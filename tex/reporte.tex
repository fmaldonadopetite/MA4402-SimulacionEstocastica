\documentclass[10pt]{article}
\usepackage[utf8]{inputenc}
\usepackage[activeacute,spanish,es-nodecimaldot]{babel}
\usepackage[left=1.5cm,top=1.5cm,right=1.5cm, bottom=1.5cm,letterpaper, includeheadfoot]{geometry}

\usepackage{amssymb, amsmath, amsthm}
\usepackage{bm}
\usepackage{dsfont}
\usepackage{graphicx}
\usepackage{lmodern,url}
\usepackage[x11names,svgnames,dvipsnames]{xcolor}
\usepackage[most]{tcolorbox}
\usepackage[colorlinks=true,urlcolor=brown!75!black!]{hyperref}

\usepackage{tikz}
\usetikzlibrary{arrows.meta}
\usepackage{wrapfig}

\usepackage{fancyhdr}
\pagestyle{fancy}
\fancypagestyle{plain}{%
\fancyhf{}
\lhead{\footnotesize\itshape\bfseries\rightmark}
\rhead{\footnotesize\itshape\bfseries\leftmark}
}

%Secciones sin numeración
\setcounter{secnumdepth}{0}

% macros
\newcommand{\QQ}{\mathbb Q}
\newcommand{\RR}{\mathbb R}
\newcommand{\NN}{\mathbb N}
\newcommand{\ZZ}{\mathbb Z}
\newcommand{\CC}{\mathbb C}

%Teoremas, Lemas, etc.
\theoremstyle{plain}
\newtheorem{teo}{Teorema}
\newtheorem{lem}{Lema}
\newtheorem{prop}{Proposición}
\newtheorem{cor}{Corolario}

\theoremstyle{definition}
\newtheorem{defi}{Definición}
\newtheorem{eje}{Ejemplo}
\newtheorem{ejer}{Ejercicio}
% fin macros

%%%%% NOMBRE Y FECHA
\newcommand{\ramo}{MA4402 - Simulación Estocástica} %Nombre y código del ramo
\newcommand{\profe}{Joaquin Fontbona T.}%Nombre del profe
\newcommand{\nmbrA}{Francisco Maldonado P.}%Nombre del estudiante A
\newcommand{\nmbrB}{Víctor Sáez M.}    %Nombre del estudiante B
\newcommand{\nmbrC}{}    %Nombre del estudiante C
\newcommand{\fecha}{22 de diciembre 2021}%Fecha
\newcommand{\titu}{Simulación de flujo monetario como fénomeno físico } %Título

%%%%%%%%%%%%%%%%%%

%Macros para este documento
\newcommand{\cin}{\operatorname{cint}}


\begin{document}
\pagestyle{empty}
%Encabezado
\fancyhead[L]{Facultad de Ciencias Físicas y Matemáticas}
\fancyhead[R]{Universidad de Chile}
\vspace*{-1.2 cm}
\begin{minipage}{0.6\textwidth}
\begin{flushleft}
\hspace*{-0.5cm}\textbf{\ramo}\\
\hspace*{-0.5cm}\textbf{Docente:} \profe\\
\hspace*{-0.5cm}\textbf{Nombre:} \nmbrA\quad \nmbrB\quad \nmbrC
\\
\hspace*{-0.5cm}\textbf{Fecha:} \fecha\\
\smallskip
\end{flushleft}
\end{minipage}
\begin{minipage}{0.36\textwidth}
\begin{flushright}
\includegraphics[scale=0.35]{fcfm}
\end{flushright}
\end{minipage}
\bigskip
%Fin encabezado

\begin{center}
\Huge\textbf{\titu}
\end{center}
\bigskip

\section{Cadena}

Se suponen $N$ agentes económicos, cada uno con su riqueza $w_{i}\geq 0 $. Sea $W = \sum\limits_{i=1}^{N} w_{i}$ la  cantidad total de riqueza. 
A partir de esto se construye la variable aleatoria $W_{i}$ correspondiente a la riqueza del agente $i$. Estamos interesados tanto en la distribución del vector $(W_{1},...,W_{n})$ como en la de las distribuciones marginales $W_{1}$ de los agentes.
La transformación siguiente normaliza la riqueza total del sistema a $1$, \[
X_{i}=\frac{W_{i}}{W}
\]
y el vector $(X_{1},...,X_{N})$ es una partición finita aleatoria del intervalo $(0,1)$, donde los  $X_{i}$ se denominan {\sl spacings} de la partición.
La siguientes observaciones son útilses y justifican este modelo simplificado de la distribución de riqueza:
\begin{itemize}
    \item Si $w_{i}$ representa riqueza, pueude ser negativa debido a deuda. En este caso, uno puede desplazar la riqueza a valores positivos sustrayendo 
    \item
\end{itemize}



\end{document}

%%%%%%%%%%%%%%%%%%%%%%%%%%%%%%%%%%%%%%%%%%%%%%%%%%%%%%%%%%%%%%%%%%%%%%%%%%%%%%%%%%%%%%%%%%%%
%%%%%%%%%%%%%%%%%%%%%%%%%%%%%%%%%%%%%%%%%%%%%%%%%%%%%%%%%%%%%%%%%%%%%%%%%%%%%%%%%%%%%%%%%%%%


