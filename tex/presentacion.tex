
%%Secciones y Frames  
    %%Las secciones en muchos temas no generan ni siquiera una diapositiva, son más como una forma de agrupar diapositivas. Los frames son las diapositivas la sintaxis es la siguiente \begin{frame}{Nombre del fame} ...... \end{frame}. 

%%Transiciones
    %Los \pause son para abrir la diapo de a poco (cada pause crea una slide distinta). Una forma de tener mayor control sobre una lista es añadiendo un número entre <> al comando \item, de esta forma el elemento de la lista sólo aparecerá en ese número de slide, si seguido al número se añade un -, el elemento aparecerá en ese número de slide y permanecerá en los siguientes, este argumento opcional <> también sirve para otros comandos que afecten el texto, por ejemplo \textbf<4->{hola}, que sólo cambiará a negrita desde la slide 4. También puede resultar útil saber que \pause[numero] hará que todo lo que venga después aparezca desde el slide número (hasta el siguiente \pause).

%%Resaltar y Bloques
    %% Para resaltar una palabra o texto importantes en la presentación es útil el comando \alert{texto importante}. Hay  tres tipos de bloques útiles para la presentación por defecto: block,alertblock,example, un ejemplo de utilización es el siguiente \begin{block}{Nombre del bloque} ...... \end{block}

%%Achicar fórmulas
    %%Para achicar fórmulas, antes del entorno mátematico introduce un comando de tamaño y después del entorno matemático reestablece el tamaño con \normalsize. Los tamaños posibles son \tiny<\scriptsize<\footnotesize<\small<\normalsize<\large<\Large<\LARGE<\huge<\Huge


%%Columnas
    %%Encontré dos formas de trabajar con columnas, columns parece más práctico pero multicols más bonito y fácil. En columns abres un entorno columns (\begin{columns}.... \end{columns} y dentro creas cada columna con \column{ancho que quieras} en general a cada columna le das una fracción del ancho del texto, por ejemplo (\column{0.5\textwidth} para crear una columna con la mitad del ancho de la diapo y \column{0.33\textwidth para una columna con un tercio de la diapo}). Para muticolumns abres un entorno multicols e indicas cuantas columnas quieres (e.g: \begin{multicols}{3}....\end{multicols}) y te reparte el texto a partes iguales entre columnas.



%% Para mayor información visitar https://www.overleaf.com/learn/latex/Beamer

%%Tipo de documento 
\documentclass{beamer}

%%Tema visual (Temas que recomiendo son Dresden, Berkeley, Frankfurt y metropolis)
\usetheme[block=fill,background=light,progressbar=frametitle]{metropolis}

%%Comando para cambiar el color del tema. Temas de color completos (en orden de mi preferencia) son: spruce,seagull, dove, wolverine, crane, beaver, monarca, beetle, fly, albatross
%\usecolortheme{seagull}

%% Paquetes
\usepackage{graphicx}
\usepackage{multicol}
\setlength{\columnseprule}{0.5pt}
\usepackage{wrapfig}
\usepackage{subcaption}

%%Macros
\newcommand{\Q}{\mathbb Q}
\newcommand{\R}{\mathbb R}
\newcommand{\N}{\mathbb N}
\newcommand{\Z}{\mathbb Z}
\newcommand{\C}{\mathbb C}


%% Info de Portada
\title{Simulación de flujo monetario como fénomeno físico }
\subtitle{Proyecto Final}
\date{22 de diciembre de 2021}
\author{Francisco Maldonado P. \and Víctor Sáez M.}%para más de un autor, separarlos con \and o \\
\institute{MA4402 - Simulación Estocástica}%Aquí pongo el ramo
\logo{\includegraphics[height=1cm]{fcfm.png}}

%%Inicio de documento
\begin{document}

%%Portada
\maketitle 

%%Tabla de Contenidos
\begin{frame}{Tabla de Contenidos}
    \tableofcontents
\end{frame}

%%Primera Sección 
\section{Preliminares Teóricos}

%%Primer Frame
\begin{frame}{Presentación de la Idea}

Se suponen $N$ agentes económicos, cada uno con su  riqueza $w_{i}\geq 0 $.\\
$W = \sum\limits_{i=1}^{N} w_{i}$ es la  cantidad total de riqueza para el conjunto de agentes. \\
A partir de esto se construye la variable aleatoria $W_{i}$ correspondiente al stock  del agente $i$. \\
%Obs: es equivalente a considerar riqueza negativa, pues basta desplazar restando a cada w_i el valor más negativo.
\end{frame}

\begin{frame}{Presentación de la Idea}
Podemos normalizar la riqueza total del sistema , $X_{i}=\frac{W_{i}}{W}$ y el vector $\mathbf{X} = (X_{1},...,X_{N})$ corresponde a una partición finita aleatoria del intervalo $(0,1)$.\\

El vector $\mathbf{X}$  vive en el simplex de dimensión $N-1$ definido por
\[
    \Delta_{N-1} := \left\{ \mathbf{x} = (x_{1},\cdots, x_{N}): x_{i}\geq 0,\forall i=1,...,N \text{ y }\sum\limits_{i=1}^{N}x_{i} = 1 \right\}
\]

\end{frame}

\begin{frame}{Modelos Estocásticos}

Definimos en base a este esquema un modelo que incorpora una evolución temporal estocástica para la distribución de riqueza de los agentes.

\end{frame}

\begin{frame}{Tiempo Discreto - Espacio Continuo}
Sea $\mathbf{X} = \mathbf{x}$ el estado actual de la variable aleatoria $\mathbf{X}$. Para cualquier par ordenado de índices $ i,j \in [1,N]$ escogidos aleatoriamente de forma uniforme, se definen las operaciones {\sl coagulación} y {\sl fragmentación}:
\begin{itemize}
    \item $coag_{ij}(\mathbf{x}): \Delta_{N-1} \rightarrow \Delta_{N-2}$  crea un nuevo agente $x= x_{i}+x_{j}$ mientras la proporción de riqueza para el resto permanece inalterado. %Simula la agregación de stocks de dos agentes en un sólo stock (como en fusiones y adquisisiones). 
    \item $frag(\mathbf{x}): \Delta_{N-2} \rightarrow \Delta_{N-1}$ toma el $x$ definido previamente y  lo divide en dos agentes  nuevamente, definiendo $x_{i} = ux$ y $x_{j} = (1-u)x$, con $u\sim U[0,1]$. %Simula la división del stock de un agente en dos o más stocks (como en herencias y fallos)
\end{itemize}
\end{frame}

\begin{frame}{Tiempo Discreto - Espacio Continuo}
En cada tiempo el estado del proceso $\mathbf{X}\in \Delta_{N-1}$ cambia de acuerdo a la composición de una coagulación con una fragmentación y la secuencia de operadores coagulación y fragmentación define una cadena de Markov homogénea en $\Delta_{N-1}$.\\
\end{frame}

\begin{frame}{Tiempo Discreto - Espacio Continuo}
Sea $\mathbf{x}(t) = (x_{1}(t),..., x_{i}(t), ..., x_{j}(t), ..., x_{N}(t))$ el estado de la cadena en el tiempo $t$ con $i,j$ los estados elegidos, entonces el estado en tiempo $t+1$ corresponde a 
\begin{align*}
    \mathbf{x}(t+1) 
    &= 
(x_{1}(t+1),...,
x_{i}(t+1) = u\cdot(x_{i}(t)+x_{j}(t)), ...,\\
    &x_{j}(t+1) = (1-u)\cdot(x_{i}(t)+x_{j}(t)) 
, ..., x_{N}(t+1))
\end{align*}

Sin embargo el kernel de este proceso es degenerado, ya que cada paso sólo afecta una medida de Lebesgue $0$ del simplex. Esto se evita al considerar un simplex discreto.
\end{frame}

\begin{frame}{Tiempo Discreto - Espacio Discreto}
Sea $N$ el número de categorías (agentes) entre los cuales se clasifican $M$ objetos (monedas o tokens). En la descripción estatística o frecuencial de este sistema, un estado es una lista $\mathbf{n} = (n_{1},..., n_{N})$ con $\sum\limits_{i=1}^{N} n_{i} = M$,  el cual entrega el número de objetos que pertenecen a cada categoría. 
\end{frame}

\begin{frame}{Tiempo Discreto - Espacio Discreto}
El estado del proceso en tiempo $t\in \N_{0}$ es denotado $\mathbf{X}(t) $ y su espacio de estados es el simplex entero escalado 
\begin{small}
\begin{align*}
S_{N-1}^{(M)} 
:&= (M\cdot \Delta_{N-1}) \cap \Z^{N}\\
&= \left\{ \mathbf{n} = (n_{1},\cdots, n_{N}): 0\leq n_{i}\leq M,\ \sum\limits_{i=1}^{N}n_{i} = M,\ n_{i}\in \N_{0} \right\}
\end{align*}
\end{small}
\[
\]
\end{frame}

\begin{frame}{Tiempo Discreto - Espacio Discreto}
En este framework, una coagulación se modela tomando un par de enteros ordenados $i,j$ aleatoriamente sin reposición de $1\leq ... \leq N$ y creando una nueva categoría con $n_{i} + n_{j}$ objetos. Una fragmentación toma esta categoría y la divide en dos nuevas categorías reetiquetadas como $i,j$ donde $n_{i}^{\prime}$ es un entero uniformemente aleatorio entre $0$ y $n_{i}+ n_{j}$ y $n_{j}^{\prime}= n_{i} + n_{j}- n_{i}^{\prime}$ .

%Formalmente con la coagulación nos movemos de espacio de estados $S_{N-1}^{(n)}$ a $S_{N-2}^{(n)}$ y con la fragmentación nos devolvemos a $S_{N-1}^{(n)}$, pero ignoramos el estado intermedio y definimos el proceso tan sólo en $S_{N-1}^{(n)}$. 
\end{frame}


\begin{frame}{Tiempo Discreto - Espacio Discreto}
Es directo que las probabilidades de transición para $\mathbf{X}$ están dadas por 
\begin{footnotesize}
 \[
     \mathbb{P} \left( \mathbf{X}(t+1) = \mathbf{n}^{\prime}|\  \mathbf{X}(t) = \mathbf{n}\right)
     = \sum\limits_{i,j:i\neq j} \left[ 
         \frac{1}{N}\frac{1}{N-1}\frac{1}{n_{i}+n_{j}+1} 
         \delta_{n_{i}+n_{j}, n_{i}^{\prime} + n_{j}^{\prime}}
         \prod\limits_{k\neq i,j} \delta_{n_{k}^{\prime},n_{k}}
     \right]
\]
\end{footnotesize}

Donde $\mathbf{n}, \mathbf{n}^{\prime}\in S_{N-1}^{(M)}$ y la notación refiere a que se suma sobre todos los pares ordenados $(i,j), i\neq j$ donde la primera coordenada indica que el índice $i$ fue escogido primero. %El modelo es simétrico en $\mathbf{n}, \mathbf{n}^{\prime}$.??
\end{frame}


\begin{frame}{Tiempo Discreto - Espacio Discreto}
%La cadena es homogénea ya que la transición ya descrita es independiente del parámetro temporal. \\
%Es además aperiódica, ya que con probabilidad positiva, durante cada paso de tiempo, la cadena puede coagular y después fragmentar al mismo estado.
%Para ver esto, considera cualquier vector $(X_{1},...., X_{N}) = (x_{1},..., x_{n})$ en el simplex, con al menos un a entrada no nula, spg $x_{1}>0$. Se selecciona el índice $i=1$ con probabilidad $N^{-1}$ y cualquier índice $j$ para la coagulación y se fragmenta en $x_{1}, x_{j}$ con probabilidad $1/(x_{1}+x_{j}+1)>0$.\\
%Finalmente es irreducible, ya que desde cualquier punto $\mathbf{X} = (x_{1},...,x_{N})$ la cadena se puede mover con probabilidad positiva a cualquiera de sus vecinos 
%{ $( (x_{1},...,x_{N}) \pm (e_{i}-e_{j}) ) \cap S_{N-1}^{(n) }$ }, i.e. a cualquier punto del simplex a una $l^{1}-$distancia de $2$ del estado actual. \\
La cadena es homogénea, aperiódica e irreducible, además como el conjunto de estados es finito, también es recurrente positiva.

Con esto concluimos que la cadena $\left\{ \mathbf{X}(t) \right\} t\in \N_{0}$ tiene una una distribución de equilibrio única. \\
\textbf{Obs:}
En \cite{Scalas} se prueba que esta distribución invariante corresponde a la distribución uniforme en $S_{N-1}^{(M)}$. 
\end{frame}

\section{Desarrollo}

\begin{frame}{Planteamiento}
    En este proyecto nos restringiremos a estudiar la forma de la distribución marginal de los agentes para el estado límite, la distribución marginal invariante para la cadena 
    \[
        \pi(x) = \lim\limits_{t\rightarrow\infty}\mathbb{P}(X_{1}(t) = x) 
    \]
    .%Recurrencia positiva entrega unicidad de la invariante y como es aperiódica la distribución de los estados converge a la dist. invariante, ¿pero qué pasa con los estados marginales?
\end{frame}
\begin{frame}{Planteamiento}
    %Tanto la mecánica estadística como la economía estudian grandes colecciones (átomos vs agentes). La ley fundamental de equilibrio en la mecánica estadística es la ley de Boltzmann-Gibbs, que establece que la distribución de probabilidad de energía e es P(e) = Cexp(-e/T), T:temperatura, C: cte normalizadora. El gran suupuesto para derivar la ley de B-G es la conservación de energía, por tanto uno podría generalizar que cualqueir cantidad conservada en un gran sistema estadístico debería tener una distribución probabilística en equilibrio.
    En particular existe una distribución conocida en mecánica estadística que buscaremos aproximar:
    \begin{itemize}
        \item Relación Física-Economía.
        \item Ley fundamental de equilibrio en mecánica estadística.
        \item Justificación: conservación cuantitativa.
    \end{itemize}
\end{frame}
\begin{frame}{Ley de Boltzmann-Gibbs}
    La distribución de equilibrio $\pi(x)$ se puede derivar de la misma forma que la función de distribución de equilibrio de la energía  en física \cite{Wannier}. \\
    Dividiendo el sistema en dos subsistemas $1$ y $2$, como suponemos la riqueza conservativa y aditiva ($x = x_{1}+ x_{2})$ y la probabilidad multiplicativa $\pi = \pi_{1}\cdot \pi_{2}$, concluimos $\pi(x_{1} + x_{2}) =\pi(x_{1})\pi(x_{2}).$\\
    La solución de esta ecuación es de la forma $\pi(x) = C\cdot e^{-m/T}$, que identificamos con la distribución de Boltzmann-Gibbs.
\end{frame}

\begin{frame}{Ley de Boltzmann-Gibbs}
    Para el caso con estados continuos, de las condiciones de normalización y riqueza promedio en el sistema
    \[
    \int\limits_{0}^{\infty}\pi(x)\, dx = 1 \quad \int\limits_{0}^{\infty} x \cdot \pi(x)\, dx = \frac{M}{N}
    \]
    Se obtiene que $C= \frac{1}{T}$ y $T = \frac{M}{N}$. Por tanto la temperatura efectiva $T$ del sistema es la riqueza promedio por agente.\\
    Para el caso con estados discretos la normalzación se calcula de acuerdo a los estados.
\end{frame}
\begin{frame}{Un primer modelo}
    
\end{frame}
\begin{frame}{Métricas}
    
\end{frame}


\section{Resultados}


\section{Conclusiones}
\begin{frame}
    Dudas: qué es un kernel degenerado, cómo pasar de la existencia y unicidad de distribución invariante para el caso marginal, qué es la simetría n,n', no entendí cómo se deriva la Boltzmann-Gibbs.\\
   Trabajo futuro: formalización teórica del caso a estados discretos.
\end{frame}
\section{Referencias}

\begin{frame}
\begin{thebibliography}{}
    \bibitem{Yakovenko} A. Dr\u{a}gulescu, V.M. Yakovenko. (2000). A stylized model for wealth distribution. 
    \bibitem{Scalas} Bertram Düring, Nicos Georgiou and Enrico Scalas. (2016) A stylized model for wealth distribution. 
    \bibitem{Wannier} G.H. Wannier. (1987) Statistical Physics.
\end{thebibliography}
\end{frame}

\maketitle
\end{document}


