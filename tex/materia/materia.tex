\documentclass[10pt]{article}
\usepackage[utf8]{inputenc}
\usepackage[activeacute,spanish,es-nodecimaldot]{babel}
\usepackage[left=1.5cm,top=1.5cm,right=1.5cm, bottom=1.5cm,letterpaper, includeheadfoot]{geometry}

\usepackage{amssymb, amsmath, amsthm}
\usepackage{bm}
\usepackage{dsfont}
\usepackage{graphicx}
\usepackage{lmodern,url}
\usepackage[x11names,svgnames,dvipsnames]{xcolor}
\usepackage[most]{tcolorbox}
\usepackage[colorlinks=true,urlcolor=brown!75!black!]{hyperref}

\usepackage{tikz}
\usetikzlibrary{arrows.meta}
\usepackage{wrapfig}

\usepackage{fancyhdr}
\pagestyle{fancy}
\fancypagestyle{plain}{%
\fancyhf{}
\lhead{\footnotesize\itshape\bfseries\rightmark}
\rhead{\footnotesize\itshape\bfseries\leftmark}
}

%Secciones sin numeración
\setcounter{secnumdepth}{0}

% macros
\newcommand{\Q}{\mathbb Q}
\newcommand{\R}{\mathbb R}
\newcommand{\N}{\mathbb N}
\newcommand{\Z}{\mathbb Z}
\newcommand{\C}{\mathbb C}

%Teoremas, Lemas, etc.
\theoremstyle{plain}
\newtheorem{teo}{Teorema}
\newtheorem{lem}{Lema}
\newtheorem{prop}{Proposición}
\newtheorem{cor}{Corolario}

\theoremstyle{definition}
\newtheorem{defi}{Definición}
\newtheorem{eje}{Ejemplo}
\newtheorem{ejer}{Ejercicio}
% fin macros

%%%%% NOMBRE Y FECHA
\newcommand{\ramo}{MA4402 - Simulación Estocástica} %Nombre y código del ramo
\newcommand{\profe}{Joaquin Fontbona T.}%Nombre del profe
\newcommand{\nmbrA}{Francisco Maldonado P.}%Nombre del estudiante A
\newcommand{\nmbrB}{Víctor Sáez M.}    %Nombre del estudiante B
\newcommand{\nmbrC}{}    %Nombre del estudiante C
\newcommand{\fecha}{22 de diciembre 2021}%Fecha
\newcommand{\titu}{Simulación de flujo monetario como fénomeno físico } %Título

%%%%%%%%%%%%%%%%%%

%Macros para este documento
\newcommand{\cin}{\operatorname{cint}}


\begin{document}
\pagestyle{empty}
%Encabezado
\fancyhead[L]{Facultad de Ciencias Físicas y Matemáticas}
\fancyhead[R]{Universidad de Chile}
\vspace*{-1.2 cm}
\begin{minipage}{0.6\textwidth}
\begin{flushleft}
\hspace*{-0.5cm}\textbf{\ramo}\\
\hspace*{-0.5cm}\textbf{Docente:} \profe\\
\hspace*{-0.5cm}\textbf{Nombre:} \nmbrA\quad \nmbrB\quad \nmbrC
\\
\hspace*{-0.5cm}\textbf{Fecha:} \fecha\\
\smallskip
\end{flushleft}
\end{minipage}
\begin{minipage}{0.36\textwidth}
\begin{flushright}
\includegraphics[scale=0.35]{fcfm}
\end{flushright}
\end{minipage}
\bigskip
%Fin encabezado

\begin{center}
\Huge\textbf{\titu}
\end{center}
\bigskip

\section{Cadena}

Se suponen $N$ agentes económicos, cada uno con su stock (e.g. riqueza) $w_{i}\geq 0 $. Sea $W = \sum\limits_{i=1}^{N} w_{i}$ la  cantidad total de riqueza para el conjunto de agentes. 
A partir de esto se construye la variable aleatoria $W_{i}$ correspondiente al stock  del agente $i$. Estamos interesados tanto en la distribución del vector $(W_{1},...,W_{n})$ como en la de las distribuciones marginales $W_{1}$ de los agentes.
La transformación siguiente normaliza la riqueza total del sistema a $1$, \[
X_{i}=\frac{W_{i}}{W}
\]
y el vector $(X_{1},...,X_{N})$ corresponde a una partición finita aleatoria del intervalo $(0,1)$, donde los  $X_{i}$ se denominan {\sl spacings} de la partición.
La siguientes observaciones son útiles y justifican este modelo simplificado de la distribución de riqueza:
\begin{itemize}
    \item Si $w_{i}$ representa riqueza, puede ser negativa debido a deuda. En este caso, uno puede desplazar la riqueza a valores positivos sustrayendo la riqueza negativa con mayor valor absoluto.
    \item  Una partición de masa es una secuencia infinita $s=(s_{1},s_{2},...)$ tal que $s_{1}\geq s_{2}\geq\cdots \geq 0$ y $\sum\limits_{i=1}^{\infty} s_{i}\leq 1$.
    \item Las particiones finitas aleatorias de intervalo se pueden corresponder a particiones de masa, ordenando los spacings y añadiendo infinitos $0$.\\
\end{itemize}

El vector $\mathbf{X} = \left( X_{1},...,X_{N} \right)$ vive en el simplex de dimensión $N-1$ definido por
\[
    \Delta_{N-1} := \left\{ \mathbf{x} = (x_{1},\cdots, x_{N}): x_{i}\geq 0,\forall i=1,...,N \text{ y }\sum\limits_{i=1}^{N}x_{i} = 1 \right\}
\]
Existen dos preguntas naturales que uno se puede plantear:
\begin{itemize}
    \item ¿Cuál es la distribución del vector $(X_{1},...,X_{N})$?
    \item ¿Cuál es la distribución de la variable aleatoria $X_{1} $, la proporción de riqueza de un individuo?
\end{itemize}
%% Falta aquí considerar el tiempo, la pregunta que estudiamos nosotros es la distrtibución de X_{1} para un tiempo \infty

Definimos en base a este esquema base un modelo que incorpora una evolución temporal estocástica para la distribución de riqueza de los agentes, en particular una cadena de Markov a tiempo discreto y estados discretos. Y nos restringiremos a estudiar (aproximar) la forma de la distribución marginal de los agentes para el estado límite.???

\subsection{Modelo tiempo discreto espacio continuo: dinámica de coagulación-fragmentación}

En cada tiempo el estado del proceso $\mathbf{X}\in \Delta_{N-1}$ cambia de acuerdo a la composición de una coagulación con una fragmentación.

Sea $\mathbf{X} = \mathbf{x}$ el estado actual de la variable aleatoria $\mathbf{X}$ para cualquier par ordenado de índices $ i,j \in [1,N]$ escogidos aleatoriamente de forma uniforme, se definen las operaciones {\sl coagulación} y {\sl fragmentación}:
\begin{itemize}
    \item $coag_{ij}(\mathbf{x}): \Delta_{N-1} \rightarrow \Delta_{N-2}$  crea un nuevo agente $x= x_{i}+x_{j}$ mientras la proporción de riqueza para el resto permanece inalterado. %Simula la agregación de stocks de dos agentes en un sólo stock (como en fusiones y adquisisiones). 
    \item $frag(\mathbf{x}): \Delta_{N-2} \rightarrow \Delta_{N-1}$ toma el $x$ definido previamente y  lo divide en dos agentes  nuevamente, definiendo $x_{i} = ux$ y $x_{j} = (1-u)x$, con $u\sim U[0,1]$. %Simula la división del stock de un agente en dos o más stocks (como en herencias y fallos)
\end{itemize}

La secuencia de operadores coagulación y fragmentación define una cadena de Markov homogénea en $\Delta_{N-1}$.\\ Sea $\mathbf{x}(t) = (x_{1}(t),..., x_{i}(t), ..., x_{j}(t), ..., x_{N}(t))$ el estado de la cadena en el tiempo $t$ con $i,j$ los estados elegidos, entonces el estado en tiempo $t+1$ corresponde a 
\[
\mathbf{x}(t+1) = 
(x_{1}(t+1),...,
x_{i}(t+1) = u\cdot(x_{i}(t)+x_{j}(t)), ..., x_{j}(t+1) = (1-u)\cdot(x_{i}(t)+x_{j}(t)) 
, ..., x_{N}(t+1))
\]
Sin embargo el kernel de este proceso es degenerado, ya que cada paso sólo afecta una medida de Lebesgue $0$ del simplex.??? Esto se evita al considerar un simplex discreto.

\subsection{Modelo tiempo discreto - espacio discreto: dinámica de coagulación-fragmentación}


Sea $N$ el número de categorías (individuos) entre los cuales se clasifican $n$ objetos (monedas o tokens). En la descripción estatística o frecuencial de este sistema, un estado es una lista $\mathbf{n} = (n_{1},..., n_{N})$ con $\sum\limits_{i=1}^{N} n_{i} = n$,  el cual entrega el número de objetos que pertenecen a cada categoría. En este framework, una coagulación se modela tomando un par de enteros ordenados $i,j$ aleatoriamente sin reposición de $1\leq ... \leq N$ y creando una nueva categoría con $n_{i} + n_{j}$ objetos. Una fragmentación toma esta categoría y la divide en dos nuevas categorías reetiquetadas como $i,j$ donde $n_{i}^{\prime}$ es un entero uniformemente aleatorio entre $0$ y $n_{i}+ n_{j}$ y $n_{j}^{\prime}= n_{i} + n_{j}- n_{i}^{\prime}$ . El estado del proceso en tiempo $t\in \N_{0}$ es denotado $\mathbf{X}(t) $ y su espacio de estados es el simplex entero escalado 
\[
S_{N-1}^{(n)} = n\Delta_{N-1} \cap \Z^{N}
\]

Formalmente con la coagulación nos movemos de espacio de estados $S_{N-1}^{(n)}$ a $S_{N-2}^{(n)}$ y con la fragmentación nos devolvemos a $S_{N-1}^{(n)}$, pero ignoramos el estado intermedio y definimos el proceso tan sólo en $S_{N-1}^{(n)}$. 
Es directo que las probabilidades de transición para $\mathbf{X}$ están dadas por 
 \[
     \mathbb{P} \left( \mathbf{X}(t+1) = \mathbf{n}^{\prime}|\  \mathbf{X}(t) = \mathbf{n}\right)
     = \sum\limits_{i,j:i\neq j} \left[ 
         \frac{1}{N}\frac{1}{N-1}\frac{1}{n_{i}+n_{j}+1} 
         \delta_{n_{i}+n_{j}, n_{i}^{\prime} + n_{j}^{\prime}}
         \prod\limits_{k\neq i,j} \delta_{n_{k}^{\prime},n_{k}}
     \right]
\]

Donde $\mathbf{n}, \mathbf{n}^{\prime}\in S_{N-1}^{(n)}$ y la notación refiere a que se suma sobre todos los pares ordenados $(i,j), i\neq j$ donde la primera coordenada indica que el índice $i$ fue escogido primero. El modelo es simétrico en $\mathbf{n}, \mathbf{n}^{\prime}$.\\
La cadena es homogénea ya que la transición ya descrita es independiente del parámetro temporal. \\
Es además aperiódica, ya que con probabilidad positiva, durante cada paso de tiempo, la cadena puede coagular y después fragmentar al mismo estado.
Para ver esto, considera cualquier vector $(X_{1},...., X_{N}) = (x_{1},..., x_{n})$ en el simplex, con al menos un a entrada no nula, spg $x_{1}>0$. Se selecciona el índice $i=1$ con probabilidad $N^{-1}$ y cualquier índice $j$ para la coagulación y se fragmenta en $x_{1}, x_{j}$ con probabilidad $1/(x_{1}+x_{j}+1)>0$.\\
Finalmente es irreducible, ya que desde cualquier punto $\mathbf{X} = (x_{1},...,x_{N})$ la cadena se puede mover con probabilidad positiva a cualquiera de sus vecinos 
{ $( (x_{1},...,x_{N}) \pm (e_{i}-e_{j}) ) \cap S_{N-1}^{(n) }$ }, i.e. a cualquier punto del simplex a una $l^{1}-$distancia de $2$ del estado actual. \\

Con esto concluimos que la cadena $\left\{ \mathbf(t) \right\} t\in \N_{0}$ tiene una una distribución de equilibrio única. En \texttt{[cite needed]} se prueba que esta distribución invariante correesponde a la distribución uniforme en $n\Delta_{N-1} \cap \Z^{\N}$


En este proyecto nos restringiremos a estudiar la forma de la distribución marginal de los agentes para el estado límite, la distribución marginal invariante.
Finitud implica recurrencia positiva, esto entrega unicidad de la invariante y como es aperiodica la distribución de los estados converge a la dist. invariante, ¿pero qué pasa con los estados marginales?



\end{document}

%%%%%%%%%%%%%%%%%%%%%%%%%%%%%%%%%%%%%%%%%%%%%%%%%%%%%%%%%%%%%%%%%%%%%%%%%%%%%%%%%%%%%%%%%%%%
%%%%%%%%%%%%%%%%%%%%%%%%%%%%%%%%%%%%%%%%%%%%%%%%%%%%%%%%%%%%%%%%%%%%%%%%%%%%%%%%%%%%%%%%%%%%


