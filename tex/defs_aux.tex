\usepackage{amsmath,amssymb}
\usepackage[utf8]{inputenc}
\usepackage[activeacute,spanish]{babel}
\usepackage{enumerate}
%\renewcommand{\labelenumi}{\normalsize\bfseries P\arabic{enumi}.}
%\renewcommand{\labelenumii}{\normalsize\bfseries (\alph{enumii})}
%\renewcommand{\labelenumiii}{\normalsize\bfseries \roman{enumiii})}

%\oddsidemargin -1.0cm
%\textwidth 18.4cm
%\topmargin -2.5cm
%\textheight 24cm

\DeclareMathOperator{\sen}{sen}
\DeclareMathOperator{\senh}{senh}
\DeclareMathOperator{\arcsen}{arcsen}
\DeclareMathOperator{\tg}{tg}
\DeclareMathOperator{\arctg}{arctg}
\DeclareMathOperator{\ctg}{ctg}
\DeclareMathOperator{\dom}{Dom}
\DeclareMathOperator{\sech}{sech}
\DeclareMathOperator{\rec}{Rec}
\DeclareMathOperator{\inte}{Int}
\DeclareMathOperator{\adh}{Adh}
\DeclareMathOperator{\fr}{Fr}
\DeclareMathOperator{\Ima}{Im}
\DeclareMathOperator{\argmin}{\text{argmin}}
\DeclareMathOperator{\co}{co}
\DeclareMathOperator{\coC}{\overline{\co}}
\DeclareMathOperator{\epi}{epi}
\DeclareMathOperator{\tr}{tr}
\DeclareMathOperator{\D}{D}
\DeclareMathOperator{\sdif}{\partial}
\let\lim=\undefined\DeclareMathOperator*{\lim}{\text{lim}}
\let\max=\undefined\DeclareMathOperator*{\max}{\text{max}}
\let\min=\undefined\DeclareMathOperator*{\min}{\text{min}}
\let\inf=\undefined\DeclareMathOperator*{\inf}{\text{inf}}

\newcommand{\enc}[3]{\Large \textbf{#1}\\ \normalsize #2\\ #3}
\newcommand{\sol}{\textbf{Soluci\'on: }}
%================ATAJO FLECHAS=============================
\newcommand{\ssi}{\Longleftrightarrow}
\newcommand{\imp}{\implies}
\newcommand{\pmi}{\Longleftarrow}
\newcommand{\contra}{\rightarrow \leftarrow}
%=========================================================

%================ATAJO CONJUNTOS===========================
\newcommand{\V}{\mathbb{V}}
\newcommand{\E}{\mathbb{E}}
\newcommand{\1}{\mathds{1}}
\newcommand{\N}{\mathbb{N}}
\newcommand{\Z}{\mathbb{Z}}
\newcommand{\Q}{\mathbb{Q}}
\newcommand{\R}{\mathbb{R}}
%\newcommand{\RR}{\mathbb{R} \times \mathbb{R}}
\newcommand{\vacio}{\varnothing}
\newcommand{\Ker}{\mathbb{K}\text{er}}
\renewcommand{\Im}{\mathbb{I}\text{m}}
\newcommand{\Rel}{\mathcal{R}}
\newcommand{\gen}[1]{\left \langle #1  \right \rangle }
\renewcommand{\S}{\mathcal{S}}
\newcommand{\I}{\mathcal{I}}
\newcommand{\RR}{\mathcal{R}}
%=========================================================


%================ATAJO DERIVADAS===========================
\newcommand{\ipartial}[2]{\dfrac{\partial #1}{\partial #2}}
\newcommand{\ider}[2]{\dfrac{d #1}{d #2}}
\newcommand{\iipartial}[2]{\dfrac{\partial^2 #1}{\partial #2^2}}
\newcommand{\iider}[2]{\dfrac{d^2 #1}{d #2^2}}
\newcommand{\ijpartial}[3]{\dfrac{\partial^2 #1}{\partial #2 \partial #3}}
%=========================================================

%================ATAJO LETRAS==============================
\newcommand{\f}{\varphi}
\newcommand{\e}{\varepsilon}
%=========================================================

%================PROBABILIDADES=============================
%\newcommand{\E}{\mathbb{E}}
\newcommand{\Var}{\mathrm{Var}}
\newcommand{\Cov}{\mathrm{Cov}}
%\newcommand{\P}{\mathbb{P}}
%=========================================================

%=================MISCELANEA===============================
\newcommand{\Def}{\overset{\cdot}{=}}
\newcommand{\nconj}[1]{\left \{  1, \ldots, #1   \right \}}
\newcommand{\p}[3]{\left #1 #3 \right #2}
\newcommand{\Vector}[1]{\left ( #1_ 1, \ldots, #1 _n\right)}
\newcommand{\ds}[1]{\displaystyle{#1}}
\newcommand{\normal}{\trianglelefteq}
\renewcommand{\P}{\mathbb{P}}
\newcommand{\Esp}{\mathbb{E}}
\newcommand{\limite}{\lim\limits_{n\to \infty}}
\newcommand{\suma}[2]{\sum\limits_{#1}^{#2}}
\newcommand{\Comp}[1]{\mathcal{O}\Par{#1}}
%=========================================================

%=================VECTORES COLUMNA=========================
\newcount\colveccount
\newcommand*\colvec[1]{
        \global\colveccount#1
        \begin{pmatrix}
        \colvecnext
}
\def\colvecnext#1{
        #1
        \global\advance\colveccount-1
        \ifnum\colveccount>0
                \\
                \expandafter\colvecnext
        \else
                \end{pmatrix}
        \fi
}

%Por ejemplo \colvec{5}{a}{b}{c}{d}{e}
%=========================================================

%=================MATRICES=================================
\newcommand{\matriz}[1]{\begin{pmatrix} #1 \end{pmatrix}}

